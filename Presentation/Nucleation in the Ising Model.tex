\documentclass{beamer}

\setbeamertemplate{navigation symbols}{}
%\setbeameroption{show notes}

\usetheme{Malmoe}
\usecolortheme{beaver}
%\logo{Abraham}
%\beamertemplatenavigationsymbolsempty



\beamersetuncovermixins{\opaqueness<1>{25}}{\opaqueness<2->{15}}
\usepackage{amssymb}
%\usepackage{float}

\usepackage{wrapfig}
\usepackage{amsmath}
\usepackage[ngerman]{babel}
\usepackage[utf8]{inputenc}
\usepackage{float}
\usepackage{graphicx}
%\usepackage{wrapfig}
\usepackage{textcomp}
\usepackage{braket}
\usepackage{bbm}
\usepackage{framed}
\usepackage{bbold}
\usepackage{colortbl}
\usepackage{color}
\usepackage{ifthen}
%\usepackage{setspace}
\newcommand{\tikzfig}[2]{
\begin{figure}[h]
\begin{center}
\input{./img/TikZ/#1.tex}
\end{center}\end{figure}
}
\newcommand{\tikzfigC}[2]{\begin{figure}[h]\begin{center}\input{./img/TikZ/#1.tex}\end{center}\caption{{#2}}\end{figure}}
\newcommand{\fig}[2]{\begin{figure}[h]\begin{center}\includegraphics[width = 0.5\textwidth]{./img/#1}\end{center}\caption{{#2}}\end{figure}}
\usepackage[T1]{fontenc}
\usepackage{amsthm}

\usepackage{bm}
\usepackage{amsbsy}

\usepackage{tikz}
\usepackage{xcolor}
	\usetikzlibrary{calc}
	\usetikzlibrary{arrows}

\usepackage{scalefnt}
\usetikzlibrary{shapes,shapes.multipart}
\usepackage{caption}
\addto\captionsngerman{
\renewcommand{\figurename}{Figure}%
\renewcommand{\tablename}{Tab.}%
}


\newcommand{\N}{\mathcal{N}}
\newcommand{\I}{\mathbbm{1}}
\newcommand{\intfin}{\int\limits_{-\infty}^{+\infty} \!}
\newcommand{\intR}{\int\limits_{\mathbbm{R}}\!}
\newcommand{\intRn}{\int\limits_{\mathbbm{R}^n}\!\di ^n}
\newcommand{\intRm}[1]{\int\limits_{\R^{#1}}\! \di^{#1}}
\newcommand{\intRR}{\int\limits_{-R}^{+R}\!}
\newcommand{\di}{\mathrm{d}}



\newcommand{\dt}[1]{\frac{\di #1}{\di t}}
\newcommand{\pdt}[1]{\frac{\partial #1}{\partial t}}
\newcommand{\pdx}[1]{\frac{\partial #1}{\partial x}}
\newcommand{\pdy}[1]{\frac{\partial #1}{\partial y}}
\newcommand{\pdz}[1]{\frac{\partial #1}{\partial z}}
\newcommand{\dd}[2][1]{\frac{\di #1}{\di #2}}
\newcommand{\pp}[2][1]{\frac{\partial #1}{\partial #2}}
\newcommand{\nk}{\text{\normalfont ,}}
\newcommand{\np}{\text{\normalfont .}}

\setlength{\parskip}{1.5ex plus0.5ex minus0.5ex}
\setlength{\parindent}{0em} 
% my commands
\sloppy 
\frenchspacing 
\raggedbottom 

\newcommand{\msc}[1]{$\mathcal{#1}$}
\newcommand{\mmc}[1]{\mathcal{#1}}
\newcommand{\R}{\mathbbm{R}}
\newcommand{\eps}{\varepsilon}
\newcommand{\phip}{\varphi}
\newcommand{\msp}[1]{$\displaystyle #1 $}
\newcommand{\mmp}[1]{\displaystyle #1 }

\newcommand{\gev}[1]{\langle \langle  #1 \rangle \rangle}

\newcommand{\Gf}{\widetilde{G}(\omega,\vec{k})}




\begin{document}
\part{ABS}
\title{Nucleation in the Ising Model}
\author{Abraham Hinteregger}
\institute{University of Vienna}
\date{17.5.2013}
\titlepage

 \setcounter{tocdepth}{4}
%\frame{\tableofcontents}



\AtBeginSection[]{
\begin{frame}
%\begin{spacing}{0}
\frametitle{Chapter} 
\tableofcontents[currentsection,currentsubsection,currentsubsubsection,hideothersubsections]
%\end{spacing}
\end{frame}
}

\section{Ising Model} 
\subsection{History}
\begin{frame}\frametitle{History} 
\begin{block}{}
\begin{itemize}%[<+->]
\item 1924: \href{http://link.springer.com/content/pdf/10.1007 BF02980577.pdf}{Ernst Ising - \textit{Beitrag zur Theorie des Ferromagnetismus}\footnote{Zeitschrift für Physik Februar–April 1925, Volume 31, Issue 1, pp 253-258 }}
\begin{quote}
''Es entsteht \ldots [durch] \ldots die Beschr"ankung der Wechselwirkung auf diejenige  benachbarter Elemente [\ldots] kein Ferromagnetismus.''
\end{quote}
\item 1936: \href{http://journals.cambridge.org/action/displayAbstract?fromPage=online\&aid=2027260}{Rudolph Peierls - \textit{On Ising's model of ferromagnetism}\footnote{Cambridge Philosophical Society 1936 / Volume 32 / Issue 03 / October}}
\begin{quote}
''[\ldots] for sufficiently low temperatures the Ising model in two [or more] dimensions shows ferromagnetism [\ldots].
\end{quote}
\end{itemize}
\end{block}
\end{frame}

\subsection{Lattice \& lattice sites}

\begin{frame}\frametitle{Lattice}
\tikzfigC{Ising1D}{Lattice in 1 dimension}
\vspace*{0.25cm}
\tikzfigC{Ising2D}{Lattice in 2 \alt<9-10>{and 3 }{}dimensions}

\end{frame}


\begin{frame}\frametitle{Lattice Sites}
\begin{itemize}
\item Each site has a state $s_i = -1\lor1$
\visible<2->{\item Assignment of states $s = (s_0, s_1, s_2,\ldots, s_{N-1})$ to the lattice sites is called a configuration
\item Therefore $2^N$ unique configurations for a lattice with N lattice sites.}
\end{itemize}
\end{frame}



%: MAYBE
%\subsection{Dynamics}
%\begin{frame}\frametitle{Dynamics}
%\begin{itemize}
%\item Transition from one configuration to another - 
%\end{itemize}
%\end{frame}

\subsection{Energy \& Monte Carlo}
\subsubsection{Energy}
\begin{frame}\frametitle{Energy}
\begin{itemize}
\item Each configuration has a corresponding energy - the Hamiltonian $H$
\visible<1->{
\begin{equation*}
H(s) = H(s_0, s_1,\ldots, s_{N-1})=
\alert<3>{ -J \sum_{\text{<i,j>}}s_i\cdot s_j}
\alert<2>{- h \sum_{\text{i}}s_i}
\end{equation*}
}
\end{itemize}
\end{frame}

\begin{frame}
\begin{equation*}
H(s)= -J \sum_{\text{<i,j>}}s_i\cdot s_j
\end{equation*}\tikzfigC{Examples}{Energy contribution (from nearest neighbor interaction) of the central lattice site}
\end{frame}

\subsubsection{Monte Carlo}
\begin{frame}\frametitle{Single Flip Monte Carlo (Metropolis- Hastings Algorithm)}
\begin{itemize}
\item Current configuration is $s_t$
\begin{itemize}
\item first configuration arbitrary
\end{itemize}
\item Flip one lattice site $\rightarrow$ $s_f$
\begin{itemize}
\item  has to be chosen randomly - suitable RNG necessary
\end{itemize}
\item Calculate energy difference $\Delta H = H(s_f) - H(s)$
\item Calculate acceptance probability $P$
\begin{equation*}
P = \operatorname{min}\left(1,e^{-\beta\cdot \Delta H}\right),\qquad \beta = 1/kT > 0 
\end{equation*}
\item Generate random number $r \in [0,1[$
\begin{itemize}
\item $r<P \rightarrow s_{t+1} = s_f$ 
\item $r>P \rightarrow s_{t+1} = s_t$ 
\end{itemize}
\end{itemize}
\end{frame}

\subsection{Temperature}
\begin{frame}\frametitle{Temperature in the Ising Model}
\begin{equation*}
P = \operatorname{min}\left(1,e^{-\beta\cdot \Delta H}\right),\qquad \beta = 1/kT > 0
\end{equation*}
\begin{itemize}
\item $\Delta H < 0 \rightarrow P = 1$
\item high temperature leads to higher acceptance probability
$\rightarrow$ unordered (low magnetization, Curie Temperature $T_c$)
\item critical temperature $T_c$ when $\left<\sum_i^N s_i\right>/N \approx 0$
\visible<2->{\begin{itemize}
\item $kT_c/J = 2.269$
\end{itemize}}
\end{itemize}
\end{frame}


\subsection{Summary}
\begin{frame}\frametitle{Summary - Ising Model}
\begin{itemize}
\item molecules on a lattice - each with with one of two possible states
\item (magnetic) moments prefer to align
\item low temperatures: ordered
\item high temperatures: disordered
\end{itemize}
\end{frame}





\section{Nucleation}
\subsection{What is Nucleation?}
\begin{frame}\frametitle{Nucleation}
\begin{itemize}
\item is a phase transformation process
\item phase transformation grows from small nucleus
\begin{exampleblock}{Examples}
\begin{itemize}
\item
cloud chamber
\item
\href{http://www.youtube.com/watch?v=pTdiTe3x0Bo}{supercooled water}
\end{itemize}
\end{exampleblock}
\end{itemize}
\end{frame}
\subsection{Homogeneous  Nucleation}\begin{frame}\frametitle{Nucleation}
\begin{itemize}
\item
Homogeneous nucleation
\begin{itemize}
\item in a uniform substance
\item no nucleation until nucleus with critical size ''appears'' (due to stochastic processes)
\item higher supersaturation leads to smaller critical radius.
\item rarely occurs in nature
\end{itemize}
\item Heterogeneous Nucleation{}
\begin{itemize}
\item begins at some preferable interface and grows from there
\item much (!) more likely
\item common in nature (freezing (in most cases), bubbles in water,...)
\end{itemize}
\end{itemize}
\end{frame}
\subsection{Heterogeneous  Nucleation}
\begin{frame}\frametitle{Nucleation}
\begin{itemize}
\item
Homogeneous nucleation
\begin{itemize}
\item in a uniform substance
\item no nucleation until nucleus with critical size ''appears'' (due to stochastic processes)
\item higher supersaturation leads to smaller critical radius.
\item rarely occurs in nature
\end{itemize}
\item Heterogeneous Nucleation{}
\begin{itemize}
\item begins at some preferable interface and grows from there
\item much (!) more likely
\item common in nature (freezing (in most cases), bubbles in water,...)
\end{itemize}
\end{itemize}
\end{frame}

\section{Nucleation in the Ising Model}
\subsection{Homogeneous Nucleation}
\begin{frame}{Homogeneous Nucleation in the Ising Model}


%: ÜBERARBEITEN
\begin{itemize}
\item Necessary modifications:\begin{itemize}
\item none\\
\end{itemize}\end{itemize}\begin{itemize}
\item Problems
\begin{itemize}
\item long time until nucleus of critical size appears 
\item inefficient to simulate billions of cycles until phase change takes place
\end{itemize}
\end{itemize}
\end{frame}
\begin{frame}
\frametitle{Cluster size}
\fig{nsize.png}{Propability of finding a cluster of size N at different times\footnote{\href{http://www.ncbi.nlm.nih.gov/pubmed/16494425}{Pan, Rappl, Chandler, Balsara: J. Phys. Chem. B 2006}}}
\end{frame}
%: NACH HOMOGENER NUKLEATION TPS EINFÜGEN
\subsection{Transition Path Sampling}
\begin{frame}\frametitle{Transition Path Sampling (TPS) - ''shooting'' method\footnote{aa\href{http://dx.doi.org/10.1063\%2F1.476378}{Dellago, Bolhuis, Chandler: Advances in Chemical Physics 123 (1998)}}}
\begin{itemize}
\item needs two stable states (A \& B)
\item path through configuration space connecting these
\item change the path a little at a random point between A and B
\item sample new path and accept if it connects A with B
\end{itemize}
\end{frame}

\begin{frame}
\frametitle{Transition Path Sampling}
\fig{tps}{First path (red), slightly changed and accepted path (blue), rejected path (green)\footnote{\href{dx.doi.org/10.1088/0953-8984/21/33/333101}{Esobedo, Borrero, Araque - J. Phys.: Condens. Matter 21 (2009)}}}
\end{frame}
\subsection{Heterogeneous Nucleation}

\begin{frame}
\frametitle{Heterogeneous Nucleation}
\begin{itemize}
\item Necessary modifications:\begin{itemize}
\item handle boundaries in heterogeneous nucleation
\begin{equation*}
H(s)= \color{gray}{ -J \sum_{\text{<i,j>}}s_i\cdot s_j - h \sum_{\text{i}}s_i}\color{black} - J_s \sum_{\text{<i,j>}}^{\text{II}}s_i\cdot s_j
\end{equation*}
\item implement walls/surfaces with fixed spins
\end{itemize}
\end{itemize}
\end{frame}






\begin{frame}\frametitle{Nucleation in and out of Pores\footnote{Page, Sear - Heterogeneous Nucleation in and out of Pores PRL 97, 065701 (2006)}}
\alt<2>{\fig{upspins}{2 phases of nucleation}}{
\begin{equation*}
H(s)= { -0.8 \sum_{\text{<i,j>}}s_i\cdot s_j - 0.05 \sum_{\text{i}}s_i}\color{gray} - 0 \sum_{\text{<i,j>}}^{\text{II}}s_i\cdot s_j,\color{black}\qquad kT = 1
\end{equation*}
\begin{itemize}
\item nucleation near surfaces $10^{12}$ times faster
\item fastest in pores
\item nucleation in 2 steps
\item diversified pore sizes lead to fastest reaction as probability of existence optimal pore size is higher
\end{itemize}}
\end{frame}


\begin{frame}\frametitle{Problems}
\begin{itemize}
\item phase transitions are rare events (with realistic values for the coupling constant, ...)
\item nonequilibrium systems therefore TPS (transition path sampling) not applicable.
\alert<2->{\visible<2->{\item $\rightarrow$ Forward Flux Sampling}}
\end{itemize}
\end{frame}


\subsection{Forward Flux Sampling}
\begin{frame}
\frametitle{Forward Flux Sampling}
\begin{itemize}
\item Similar to TIS (transition interface sampling - a modified TPS)\begin{itemize}
\item initial state A: $\lambda < \lambda_A = \lambda_0$
\item final state B:\quad\!\!\! $\lambda > \lambda_B = \lambda_n$
\item path has to pass every $\lambda_i$ in increasing order (can go backwards in between too) until it reaches $\lambda_n$ (B)
\end{itemize}
\item after reaching a new interface ($\lambda_{i+1}$) configuration is stored
\item stored configurations used as starting point for new trial runs
\item trial runs continued until path reaches A ($\rightarrow$ failure) or a new interface $\lambda_{i+1}$ ($\rightarrow$ success)
\end{itemize}
\end{frame}

\begin{frame}
\frametitle{(Direct) Forward Flux Sampling\footnote{\href{iopscience.iop.org/0953-8984/21/46/463102}{Allen, Valeriani, Rein ten Wolde: 2009 J. Phys.: Condens. Matter 21}}}
\only<1>{\fig{DFFS1}{Sampling path starting in A - store configurations where the path leaves A (X)}}
\only<2>{\fig{DFFS2}{Sampling new paths from every stored configuration. Discard if path goes back to A}}
\end{frame}


%!RATE CONSTANTS

\section{Outlook}\subsection{Limitations of the Ising Model}
\begin{frame}
\frametitle{Limitations of the Ising Model}
\begin{itemize}
\item only nearest neighbor interaction (can be changed to e.g. next-nearest neighbor interaction)
\item no forces from the outside (can be changed with additional term in the Hamiltonian)
\item $\rightarrow$ more qualitative than quantitative results
\end{itemize}
\end{frame}
\subsection{Potts Model}
\begin{frame}
\frametitle{Potts Model}
\begin{itemize}
\item states not only $-1 \land 1$ but (discrete) angles.
\item $H = -J_c \sum_{i,j}cos\left(\theta_i -\theta_j\right) +  \ldots$
\item Applications
\begin{itemize}
\item
percolation (Wu: ''Percolation and the Potts Model'' (1978))
\item 
flow of foam (Sanyal, Soma: ''Viscous instabilities in flowing foams'' (2006))
\item 
cancerous tumors (Sun, Chang, Cai: ''A Discrete Simulation of Tumor Growth Concerning Nutrient Concentration'' (2004))
\end{itemize}
\end{itemize}
\end{frame}

\section{}
\subsection{Additional Literature}
\begin{frame}
\begin{itemize}
\item Page, Sear - Heterogenous Nucleation in and out of Pores (2006): PRL 97, 065701
\item Allen, Valeriani, Rein ten Wolde - Forward Flux Sampling for rare event simulations (2009): J. Phys.: Condens. Matter 21 (2009) 463102 (21pp)
\item Allen, Frenkel, Rein ten Wolde - Forward Flux Sampling-type schemes for simulating rare events: Efficiency analysis (2008): http://arxiv.org/abs/cond-mat/0602269v1
\item Escobedo, Borrero, Araque -  Transition path sampling and forward flux sampling. Applications to biological systems 2009 J. Phys.: Condens. Matter 21 333101
\end{itemize}
\end{frame}
\subsection{Simulation}
\begin{frame}
\begin{itemize}
\item Sourcefiles and binaries on my github \href{https://github.com/oerpli/Ising2D}{https://github.com/oerpli/Ising2D}
\end{itemize}
\end{frame}


\end{document}